\subsection*{Question 2.8}
\subsubsection*{1)}
Below in figure \ref{fig:q28decisiontree} can be seen the decision
tree for the situation.

% Sets the branch height.
%\branchheight{10em}
\begin{figure}[!htbp]
  \centering
  \synttree[\fbox{{}\quad{}} [No operation, $U(3)$\quad {}][\setlength{\unitlength}{0.5mm} \begin{picture}(1, 1) \put(-1,3){\circle{8}} \end{picture}[Live, $p=\frac{7}{10}$, $U(12)$][Die, $p = \frac{3}{10}$, $U(0)$]]]
  \caption{}
  \label{fig:q28decisiontree}
\end{figure}

\subsubsection*{2)}
The operation should be prefered if $U(3) < 0.7$, this is because in
order to obtain $U(12) = 1.0$, we need to have the operation. And as
there is $p(survive operation) = 0.7$, this results in $1.0 * 0.7 =
0.7$. Whereas the contrary $p(not survive operation) = 1 - p(survive
operation) = 1 - 0.7 = 0.3$, with $U(0) = 0.0$ we get $0.3 * 0.0 =
0.0$

\subsubsection*{3)}
We are to explain the calculations. It is simply a using Bayes theorem
to derive the posterior probability $p(\textrm{survive}|\textrm{pos})$
of the patient surviving given that the test was positive.

\subsubsection*{4)}
Yes, Dr. No should perform the operation, the chances of survival are
as given by the posterior probability in $3)$. Although there are
always a chance that it will fail and the patient will die.

\subsection*{Question 2.9}

The loss matrix describes the cost for each misclassification.  In the
given matrix, a misclassification that diagnoses a healthy patient as
having cancer costs 1 (just like when no cost matrix is used), but a
cancerous patient classified as healthy has a cost of 1000.  In this
way, many healthy patients may be misdiagnosed, but few cancerous
patients are misdiagnosed; just as one would expect.  The cost of
correctly diagnosing a patient, cancerous or not, is of course set to
$0$ in the matrix.

Figure \ref{fig:q29} shows the prediction trend when the loss matrix
form the assignment is used.  As expected, the cancer predictions now
outweigh the healthy predictions by far.  The overall trend is the
same as before, i.e. the the healthy predictions are mainly in the
lower right corner, but also in the lower left corner.  The border
between the predictions are, however, moved away from the cancer
predictions, causing an abundance of cancer predictions, just as we
would expect.

\begin{figure}[!htbp]
  \centering
  \includegraphics[width=0.6\textwidth]{./images/q209.pdf}
  \caption{The prediction trend when the loss matrix is used. Red dots
    are cancer predictions; green are healthy.}
  \label{fig:q29}
\end{figure}

\subsection*{Question 4.3}

We are to run the program \textit{matrixfactorization.m} to learn the
$\mathbf{U}$ and $\mathbf{V}$ and report the training set and cross
validation RMSE for different values of $K$ and $\lambda$. The result
of these runs can be seen in figure \ref{fig:q43RMSEtrte}. I seem to
have an plotting error or something else wrong with this, as it can be
seen from the figure, the RMSE for test using $K = 5$ is always lower
than using more latent dimension which does not really makes sense,
according to the training set, the ones using higher latent dimensions
should produce lower RMSE. If the RMSE for the test had been in the
reverse order I would be more happy.

\begin{figure}[!htbp]
  \centering
  \subfloat[\label{subfig:1}Training and test set 1.]{\includegraphics[width=0.45\textwidth]{./images/q43RMSEtrte1}} \hspace{0.5cm}
  \subfloat[\label{subfig:2}Training and test set 2.]{\includegraphics[width=0.45\textwidth]{./images/q43RMSEtrte2}} \\
  \subfloat[\label{subfig:3}Training and test set 3.]{\includegraphics[width=0.45\textwidth]{./images/q43RMSEtrte3}} \hspace{0.5cm}
  \subfloat[\label{subfig:4}Training and test set 4.]{\includegraphics[width=0.45\textwidth]{./images/q43RMSEtrte4}} \\
  \subfloat[\label{subfig:5}Training and test set 5.]{\includegraphics[width=0.45\textwidth]{./images/q43RMSEtrte5}} \hspace{0.5cm}
  \subfloat[\label{subfig:mean}Training and test set cross validation.]{\includegraphics[width=0.45\textwidth]{./images/q43RMSEmean}} \\
  \caption{RMSE for the five training and test sets and cross validation.}
  \label{fig:q43RMSEtrte}
\end{figure}

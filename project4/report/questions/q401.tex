\subsection*{Question 4.1}

We are to investigate how well we can predict ratings using regression
on features approach, which looks like the following:
\begin{equation*}
r_{mn} = y(\mathbf{w}, \mathbf{x}_{nm}) + \epsilon_{nm}
\end{equation*}

We are to select a number of features and base our model on this, the features I have selected can be seen in figure \ref{fig:q41features}.

\begin{figure}[!htbp]
\centering
\begin{tabular}{r|r}
Feature & Description \\
\hline
user\_age & User age \\
user\_gender & User gender \\
user\_zipcode & User zipcode \\
user\_occupation & User occupation \\
movie\_genre & Movie genre \\
timestamp & Time of the movie rating
\end{tabular}
\caption{Selected features from users and movies}
\label{fig:q41features}
\end{figure}

The following reasons apply to selecting the different features.
\begin{itemize}
\item User age is selected because it is very like that users of
  different age have different taste and therefore rate the movies in
  some pattern based on their age.
\item User gender is selected because males and females tends to have
  somewhats different tastes in movies.
\item User zipcode it is possible that users living in the same area
  will be likely to favor the same type of movies, though it is pretty
  unlikely, but I have chosen to include it anyhow.
\item User occupation could be linked to the occupation of the
  user. For example it is likely that programmer and technicians rate
  movies similarily and rather different from librarians and such.
\item Movie genre is selected because users tends to have a specific
  taste in movie genre, and therefore users with similar features
  might have a tendency to favor the same movie genre and thereby rate
  them in a similar manner.
\item Timestamp I have no real argument for selecting timestamp, it is
  very unlikely that the time of rating has anything to do with the
  rating of a given movie.
\end{itemize}

I have used a simple linear basis function model of the form seen in Bishops p. 138 eq. 3.3
\begin{equation*}
y(\mathbf{x},\mathbf{w}) = \sum_{j=0}^{M-1} w_{j}\phi_{j}(\mathbf{x}) = \mathbf{w}^{T}\boldsymbol{\phi}(\mathbf{x})
\end{equation*}

I have chosen to use the identity function as basis function $\phi(x)
= x$. Using this I have obtain the results seen in table
\ref{tab:q41RMSEtr}--\ref{tab:q41RMSEte}.

\begin{table}[!htbp]
  \begin{minipage}[b]{0.5\linewidth}
    \centering
    \begin{tabular}{c|c}
      Training set & RMSE \\
      \hline
      1 & 1.100913 \\
      2 & 1.105745 \\
      3 & 1.110072 \\
      4 & 1.110102 \\
      5 & 1.109029 \\
      \hline
      Mean & 1.107172 \\
    \end{tabular}
    \caption{RMSE on training sets.}
    \label{tab:q41RMSEtr}
  \end{minipage}
  \hspace{0.5cm}
  \begin{minipage}[b]{0.5\linewidth}
    \centering
    \begin{tabular}{c|c}
      Test set & RMSE \\
      \hline
      1 & 1.132595 \\
      2 & 1.113342 \\
      3 & 1.096032 \\
      4 & 1.095797 \\
      5 & 1.100197 \\
      \hline
      Mean & 1.107593 \\
    \end{tabular}
    \caption{RMSE on test sets.}
    \label{tab:q41RMSEte}
  \end{minipage}
\end{table}

When training the five different model we obtain five different sets
of $\mathbf{w}_{ML}$, I have then chosen also to try and mean the five
returned weights and rerunning against all the test sets. Using this
method I obtain the results reported in table
\ref{tab:q41meanwml}. Using simple mean of each rated movies I obtain
the results in table \ref{tab:q41meanrated}

\begin{table}[!htbp]
  \begin{minipage}[b]{0.5\linewidth}
    \centering
    \begin{tabular}{c|c}
      Test set & RMSE \\
      \hline
      1 & 1.131796 \\
      2 & 1.112976 \\
      3 & 1.095571 \\
      4 & 1.095538 \\
      5 & 1.099822 \\
      \hline
      Mean & 1.107141 \\
    \end{tabular}
    \caption{RMSE on test sets using the mean of the five $\mathbf{w}_{ML}$.}
    \label{tab:q41meanwml}
  \end{minipage}
  \hspace{0.5cm}
  \begin{minipage}[b]{0.5\linewidth}
    \centering
    \begin{tabular}{c|c}
      Test set & RMSE \\
      \hline
      1 & 1.036062 \\
      2 & 1.031429 \\
      3 & 1.024231 \\
      4 & 1.019363 \\
      5 & 1.028093 \\
      \hline
      Mean & 1.027836 \\
    \end{tabular}
    \caption{RMSE on test sets using mean of each rated movie within training set.}
    \label{tab:q41meanrated}
  \end{minipage}
\end{table}

As it can be seen from table \ref{tab:q41meanwml} the RMSE obtained is
lower than the mean the five individual model, this also correspons
with the intuitions, the more data one uses to train the model the
more accurate it becomes. From table \ref{tab:q41meanrated} it can be
seen that simple using the mean value of each rated movie obtains
better results than the simple linear model. I have chosen to set
movies with no rating to $0$, and improvement would probably be to set
all unrated movies to the mean value of all rated movies. But this
have not been tested.

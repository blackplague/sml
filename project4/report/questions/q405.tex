\subsection*{Question 4.5}

We are first to perform PCA on the rows of the matrix $\mathbf{Y}$,
this is the rows representing the users. Then we are to look for
patterns, by annotating the plot with different kind of information
for example user gender and such things.

<Missing information>


From figure \ref{fig:q45starwars} we see Starwars (1977) plotted along
with it ten nearest neightbours using $k = 50$ and $\lambda = 5$

\begin{figure}[!htbp]
  \centering
  \includegraphics[width=0.95\textwidth]{./images/q45starwars}
  \caption{Shows Starwars (1977) along with its 10 nearest neighbours
    and other movies. Starwars is annotated by the large red spot.}
  \label{fig:q45starwars}
\end{figure}

When performing the movie centric visualization of the data by
performing PCA on the rows of $\mathbf{V}$ a really clear pattern is
found by plotting the all the movies and coloring them according to
their mean rating, defined as the sum of all non-zero entries divided
by the count of non-zero entries. The result of this plotting can be seen in figure \ref{fig:q45meanrating}

\begin{figure}[!htbp]
  \centering
  \includegraphics[width=0.95\textwidth]{./images/q45meanrating}
  \caption{Show all the movies colored according to their mean rating,
    it is easy to see a clear pattern, where the movies are divided
    into four group. And movies with a mean rating of $> 4.5$ is
    scattered all over around the center.}
  \label{fig:q45meanrating}
\end{figure}

We will now be looking a little more close at the movies which have a
rating of $> 4.5$ to see if we can figure out what characterizes them.
The plot of only these movies can be seen in figure \ref{fig:q45supermovies}

\begin{figure}[!htbp]
  \centering
  \includegraphics[width=0.95\textwidth]{./images/q45supermovies}
  \caption{Shows the movies scoring the highest mean rating, along
    with their genre and number of rating per movie.}
  \label{fig:q45supermovies}
\end{figure}

It is apperent that the reason why these movies have a high mean
rating is the fact that there are almost no rating per movies, most of
them have only one rating, and if this is a five, then it is bound to
be amongst the top mean rated. Using a weighting scheme where also the
amount of user rating per movie are taking into account would seem
mostly appropriate.

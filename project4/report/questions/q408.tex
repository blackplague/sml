\subsection*{Question 4.8}

The code for setting up the Google matrix is very simple, and could
probably be made even simpler than it is already. The Google matrix is
setup as follows

\begin{equation*}
  \mathbf{G} =
  \left[ \begin{array}{c|c}
      \mathbf{0} & \mathbf{R} \\
      \hline
      \mathbf{R}^{\textrm{T}} & \mathbf{0}
\end{array} \right]      
\end{equation*}

Where the $\mathbf{R}$ is modified to show only ones and zeros, one if
there is a connection between a user and movie. This is done in MatLab
as shown in listing \ref{lst:googlematrix}.

\begin{lstlisting}[caption={Creation of the Google matrix in Matlab}, captionpos=b,
    label={lst:googlematrix}, float=h, numbers=none]
  % Read in the rating matrix R given the path.
  [ R, ~, ~, ~, ~ ] = ...
    readmovielens('ml-data/', path );
  
  % Init sizes.
  userSize = size(R,1); movieSize = size(R,2);
  % Initialize G to zeros.
  G = zeros(userSize+movieSize,userSize+movieSize);
  
  % Fill in ones where the R matrix entries are r > 0
  G(1:userSize,userSize+1:end) = R>0;
  G(userSize+1:end,1:userSize) = (R>0)';
\end{lstlisting}

To be able to scale up to www size it is necessary to use a sparse
data structure. There are various ways of representing sparse
matrices, below are listed the most common ones.
\begin{itemize}
\item DOK - Dictionary of Keys is commonly used to construct matrices,
  afterwards it is converted to another type that supports efficient
  sparse matrix arithmetic.
\item LIL - List of Lists this is fast for incremental construction of
  matrices. The entries are usually sorted according to column index
  to allow faster lookup.
\item COO - Coordinate List here the values of the matrix is stored in
  the form $(\mathit{row}, \mathit{column}, \mathit{value})$, the
  entries are usually sorted to allow faster random access. This is
  also the standard way of defining a sparse matrix within MatLab.
\item CSR - Compressed Sparse Row here the data is stored as follows
  $(\mathit{value}, \mathit{column\_idx}, \mathit{row\_pointer})$ this
  format is efficient for arithmetic operations, row slicing and
  matrix-vector products.
\item CSC - Same as CSR except its now an row index stored in second
  place and a column pointer stored last, this results in fast
  arithmetic operations, column slicing and matrix-vector products.
\end{itemize}
